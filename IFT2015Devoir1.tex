\documentclass[10pt]{report}
\usepackage[utopia]{mathdesign} 
%\usepackage{amsmath,amsfonts,amsthm,amssymb,mathtools}

\usepackage[french]{babel}
%\usepackage[utopia]{mathdesign}

% Permet d'ajuster la taille des marges et de la distance pour les footer
\usepackage[tmargin=2cm,rmargin=0.4in,lmargin=0.4in,bmargin=2cm,footskip=.2in]{geometry}

% Permet d'optimiser l'affichage de différents symboles et formules mathématiques
%\usepackage{amsmath,amsthm,mathtools}
%usepackage{amsmath,amsfonts,amsthm,amssymb,mathtools}

\usepackage{svg}
% Modifie l'apparence des nombre en mathmode et textmode
%\usepackage[varbb]{newpxmath}

% Modifier l'apparence des fractions
\usepackage{xfrac}

            %%%%%%%%%%%%%%%%%  Sect.        14 Oct 2024     %%%%%%%%%%%%%%%%%%%%%%%%%%%%%%%%%%%%%%%%%%%%%%%%%%%%%%%%%%%
\usepackage{graphicx}
\usepackage{caption}
\usepackage{subcaption}
\usepackage{arydshln}
            %%%%%%%%%%%%%%%%%  Sect.        14 Oct 2024     %%%%%%%%%%%%%%%%%%%%%%%%%%%%%%%%%%%%%%%%%%%%%%%%%%%%%%%%%%%
\usepackage{balance}
\usepackage{dirtree}
\usepackage{titlesec}






% Permet de rayer (barrer) l'argument avec la touche
% \cancel{} \bcancel{} ou \xcancel{}
\usepackage[makeroom]{cancel}

% Extension du package amsmath; corrige certains bugs et déficiences de son prédecesseur
%\usepackage{mathtools}

% This package provides most of the flexibility you may want to customize the three basic list
% environments (enumerate, itemize and description)
\usepackage{bookmark} 

% Réorganiser les théorèmes et Lemmes. Usage complexe. 
% Référence : https://ctan.math.illinois.edu/macros/latex/contrib/theoremref/theoremref-doc.pdf
\hypersetup{hidelinks}
\usepackage{hyperref,theoremref} 

% Fournit un environnement pour créer des boîtes colorées
\usepackage[most,many,breakable]{tcolorbox}


%\newcommand\mycommfont[1]{\footnotesize\ttfamily\textcolor{blue}{#1}}\SetCommentSty{mycommfont}

%\newcommand{\incfig}[1]{%\def\svgwidth{\columnwidth}\import{./figures/}{#1.pdf_tex}}
\newcommand{\arc}[1]{\wideparen{#1}}

%Pour colorer les lignes séparatrices de tableaux
\usepackage{colortbl}
\usepackage{tikzsymbols}

\usepackage{framed}
\usepackage{titletoc}
\usepackage{etoolbox}
\usepackage{lmodern}
\usepackage{tabularx}
\usepackage{enumitem}
\usepackage{amsthm}
            %%%%%%%%%%%%%%%%%  Sect.        14 Oct 2024     %%%%%%%%%%%%%%%%%%%%%%%%%%%%%%%%%%%%%%%%%%%%%%%%%%%%%%%%%%%

\usepackage{lipsum}
\usepackage{titling}
\renewcommand\maketitlehooka{\null\mbox{}\vfill}
\renewcommand\maketitlehookd{\vfill\null}

\newcommand{\varitem}[3][black]{%
  \item[%
   \colorbox{#2}{\textcolor{#1}{\makebox(5.5,7){#3}}}%
  ]
}
\usepackage{afterpage}
\newcommand\myemptypage{
    \null
    \thispagestyle{empty}
    \addtocounter{page}{-1}
    \newpage
    }




% from https://tex.stackexchange.com/a/167024/121799
\newcommand{\ClaudioList}{red,DarkOrange1,Goldenrod1,Green3,blue!50!cyan,DarkOrchid2}
\newcommand{\SebastianoItem}[1]{\foreach \X[count=\Y] in \ClaudioList
{\ifnum\Y=#1\relax
\xdef\SebastianoColor{\X}
\fi
}
\tikz[baseline=(SebastianoItem.base),remember
picture]{%
\node[fill=\SebastianoColor,inner sep=4pt,font=\sffamily,fill opacity=0.5] (SebastianoItem){#1)};}
}
\newcommand{\SebastianoHighlight}{\tikz[overlay,remember picture]{%
\fill[\SebastianoColor,fill opacity=0.5] ([yshift=4pt,xshift=-\pgflinewidth]SebastianoItem.east) -- ++(4pt,-4pt)
-- ++(-4pt,-4pt) -- cycle;
}}   
            %%%%%%%%%%%%%%%%%  Sect.        14 Oct 2024     %%%%%%%%%%%%%%%%%%%%%%%%%%%%%%%%%%%%%%%%%%%%%%%%%%%%%%%%%%%





%====================================================================

%====================================================================
\newcommand*{\authorimg}[1]%
    { \raisebox{-1\baselineskip}{\includegraphics[width=\imagesize]{#1}}}
\newlength\imagesize  

\usepackage{pgfplots}
\pgfplotsset{compat=1.17}

%==========================================================================================
\usepackage{libris} 
\usepackage{etoolbox}
\usepackage[export]{adjustbox}% for positioning figures

\makeatletter
% Force le chapitre suivant sur la ligne succedant la fin du 
% chapitre précédent
\patchcmd{\chapter}{\if@openright\cleardoublepage\else\clearpage\fi}{}{}{}
\makeatother
\usepackage[Sonny]{fncychap}


%boîte de couleur grise
\tcbset{
  graybox/.style={
    colback=gray!20,
    colframe=black,
    sharp corners=downhill,
    boxrule=1pt,
    left=5pt,
    right=5pt,
    top=5pt,
    bottom=5pt,
    boxsep=0pt,
	 % <-- add four values for each corner
  }
}
\newtcolorbox{graybox}{graybox}

%==========================================================================================



\usepackage{xcolor}
\usepackage{varwidth}
\usepackage{varwidth}
\usepackage{etoolbox}
%\usepackage{authblk}
\usepackage{nameref}
\usepackage{multicol,array}
\usepackage{tikz-cd}
\usepackage[ruled,linesnumbered,ruled]{algorithm2e}
\usepackage{comment} % enables the use of multi-line comments (\ifx \fi) 
\usepackage{import}
\usepackage{xifthen}
\usepackage{pdfpages}
\usepackage{transparent}


%\usepackage[french]{babel}
\usepackage{listings} % pour écrire du code dans un environnement
\lstset{
  basicstyle=\ttfamily,
  columns=fullflexible,
  keepspaces=true
}
\usepackage{caption}
\usepackage{float} % Pour forcer les images au bon endroit



\usepackage[T1]{fontenc}
\usepackage{csquotes}
%%%%%%%%%%%%%%%%%%%%%%%%%%%%%%%%%%%%%%%%%%%%%%%%%%%%%%%%%%%%%%%%%%%%%%%%%%%%%%%%%%%%%%%%%%%%%%%%%
%									ENSEMBLE DE COULEURS
%%%%%%%%%%%%%%%%%%%%%%%%%%%%%%%%%%%%%%%%%%%%%%%%%%%%%%%%%%%%%%%%%%%%%%%%%%%%%%%%%%%%%%%%%%%%%%%%%

\definecolor{myg}{RGB}{56, 140, 70}
\definecolor{myb}{RGB}{45, 111, 177}

\definecolor{mygbg}{RGB}{235, 253, 241}


\definecolor{myr}{RGB}{199, 68, 64}
\definecolor{mytheorembg}{HTML}{F2F2F9}
\definecolor{mytheoremfr}{HTML}{00007B}
\definecolor{mylenmabg}{HTML}{FFFAF8}
\definecolor{mylenmafr}{HTML}{983b0f}
\definecolor{mypropbg}{HTML}{f2fbfc}
\definecolor{mypropfr}{HTML}{191971}
\definecolor{myexamplebg}{HTML}{F2FBF8}
\definecolor{myexamplefr}{HTML}{88D6D1}
\definecolor{myexampleti}{HTML}{2A7F7F}
\definecolor{mydefinitbg}{HTML}{E5E5FF}
\definecolor{mydefinitfr}{HTML}{3F3FA3}
\definecolor{notesgreen}{RGB}{0,162,0}
\definecolor{myp}{RGB}{197, 92, 212}
\definecolor{mygr}{HTML}{2C3338}
\definecolor{myred}{RGB}{127,0,0}
\definecolor{myyellow}{RGB}{169,121,69}
\definecolor{myexercisebg}{HTML}{F2FBF8}
\definecolor{myexercisefg}{HTML}{88D6D1}
\definecolor{myred}{RGB}{127,0,0}
\definecolor{myyellow}{RGB}{169,121,69}
\definecolor{LightLavender}{HTML}{DFC5FE}

\definecolor{blue}{HTML}{008ED7}
\definecolor{mygray}{gray}{0.75}
\definecolor{lightBlue}{RGB}{235, 245, 255}
\definecolor{tcbcolred}{RGB}{255,0,0}
\definecolor{myGreen}{HTML}{009900}

% command to circle a text
\newtcbox{\entoure}[1][red]{on line,
	arc=3pt,colback=#1!10!white,colframe=#1!50!black,
	before upper={\rule[-3pt]{0pt}{10pt}},boxrule=1pt,
	boxsep=0pt,left=2pt,right=2pt,top=1pt,bottom=.5pt}
% command for the circle for the number of part entries
\newcommand\Circle[1]{\tikz[overlay,remember picture]
	\node[draw,circle, text width=18pt,line width=1pt] {#1};}

\newtcbox{\entouree}[1][red]{on line,
	arc=3pt,colback=#1!10!white,colframe=#1!50!white,
	before upper={\rule[-3pt]{0pt}{10pt}},boxrule=1pt,
	boxsep=0pt,left=2pt,right=2pt,top=1pt,bottom=.5pt}

\newcommand{\shellcmd}[1]{\\\indent\indent\texttt{\footnotesize\# #1}\\}

%=====================================================================

\patchcmd{\tableofcontents}{\contentsname}{\rmfamily\contentsname}{}{}
% patching of \@part to typeset the part number inside a framed box in its own line
% and to add color
\makeatletter
\patchcmd{\@part}
  {\addcontentsline{toc}{part}{\thepart\hspace{1em}#1}}
  {\addtocontents{toc}{\protect\addvspace{20pt}}
    \addcontentsline{toc}{part}{\huge{\protect\color{myyellow}%
      \setlength\fboxrule{2pt}\protect\Circle{%
        \hfil\thepart\hfil%
      }%
    }\\[2ex]\color{myred}\rmfamily#1}}{}{}

%\patchcmd{\@part}
%  {\addcontentsline{toc}{part}{\thepart\hspace{1em}#1}}
%  {\addtocontents{toc}{\protect\addvspace{20pt}}
%    \addcontentsline{toc}{part}{\huge{\protect\color{myyellow}%
%      \setlength\fboxrule{2pt}\protect\fbox{\protect\parbox[c][1em][c]{1.5em}{%
%        \hfil\thepart\hfil%
%      }}%
%    }\\[2ex]\color{myred}\sffamily#1}}{}{}
\makeatother
% this is the environment used to typeset the chapter entries in the ToC
% it is a modification of the leftbar environment of the framed package
\renewenvironment{leftbar}
  {\def\FrameCommand{\hspace{6em}%
    {\color{myyellow}\vrule width 2pt depth 6pt}\hspace{1em}}%
    \MakeFramed{\parshape 1 0cm \dimexpr\textwidth-6em\relax\FrameRestore}\vskip2pt%
  }
 {\endMakeFramed}

% using titletoc we redefine the ToC entries for parts, chapters, sections, and subsections
\titlecontents{part}
  [0em]{\centering}
  {\contentslabel}
  {}{}
\titlecontents{chapter}
  [0em]{\vspace*{2\baselineskip}}
  {\parbox{4.5em}{%
    \hfill\Huge\rmfamily\bfseries\color{myred}\thecontentspage}%
   \vspace*{-2.3\baselineskip}\leftbar\textsc{\small\chaptername~\thecontentslabel}\\\rmfamily}
  {}{\endleftbar}
\titlecontents{section}
  [8.4em]
  {\rmfamily\contentslabel{3em}}{}{}
  {\hspace{0.5em}\nobreak\color{myred}\normalfont\contentspage}
\titlecontents{subsection}
  [8.4em]
  {\rmfamily\contentslabel{3em}}{}{}  
  {\hspace{0.5em}\nobreak\color{myred}\contentspage}


\tcbset{
  tbcsetLavender/.style={
    on line, 
    boxsep=4pt, left=0pt,right=0pt,top=0pt,bottom=0pt,
    colframe=white, colback=LightLavender,  
    highlight math style={enhanced}
  }
}
\tcbset{
  grayb/.style={
    on line, 
    boxsep=4pt, left=0pt,right=0pt,top=0pt,bottom=0pt,
    colframe=white, colback=gray!30,  
    highlight math style={enhanced}
  }
}


%==========================================================================

%PYTHON LSTLISTING STYLE

% Define colors
\definecolor{Pgruvbox-bg}{HTML}{282828}
\definecolor{Pgruvbox-fg}{HTML}{ebdbb2}
\definecolor{Pgruvbox-red}{HTML}{fb4934}
\definecolor{Pgruvbox-green}{HTML}{b8bb26}
\definecolor{Pgruvbox-yellow}{HTML}{fabd2f}
\definecolor{Pgruvbox-blue}{HTML}{83a598}
\definecolor{Pgruvbox-purple}{HTML}{d3869b}
\definecolor{Pgruvbox-aqua}{HTML}{8ec07c}

% Define Python style
\lstdefinestyle{PythonGruvbox}{
	language=Python,
	identifierstyle=\color{lst-fg},
	basicstyle=\ttfamily\color{Pgruvbox-fg},
	keywordstyle=\color{Pgruvbox-yellow},
	keywordstyle=[2]\color{Pgruvbox-blue},
	stringstyle=\color{Pgruvbox-green},
	commentstyle=\color{Pgruvbox-aqua},
	backgroundcolor=\color{Pgruvbox-bg},
	%frame=tb,
	rulecolor=\color{Pgruvbox-fg},
	showstringspaces=false,
	keepspaces=true,
	captionpos=b,
	breaklines=true,
	tabsize=4,
	showspaces=false,
	numbers=left,
	numbersep=5pt,
	numberstyle=\tiny\color{gray},
	showtabs=false,
	columns=fullflexible,
	morekeywords={True,False,None},
	morekeywords=[2]{and,as,assert,break,class,continue,def,del,elif,else,except,exec,finally,for,from,global,if,import,in,is,lambda,nonlocal,not,or,pass,print,raise,return,try,while,with,yield},
	morecomment=[s]{"""}{"""},
	morecomment=[s]{'''}{'''},
	morecomment=[l]{\#},
	morestring=[b]",
	morestring=[b]',
	literate=
	{0}{{\textcolor{Pgruvbox-purple}{0}}}{1}
	{1}{{\textcolor{Pgruvbox-purple}{1}}}{1}
	{2}{{\textcolor{Pgruvbox-purple}{2}}}{1}
	{3}{{\textcolor{Pgruvbox-purple}{3}}}{1}
	{4}{{\textcolor{Pgruvbox-purple}{4}}}{1}
	{5}{{\textcolor{Pgruvbox-purple}{5}}}{1}
	{6}{{\textcolor{Pgruvbox-purple}{6}}}{1}
	{7}{{\textcolor{Pgruvbox-purple}{7}}}{1}
	{8}{{\textcolor{Pgruvbox-purple}{8}}}{1}
	{9}{{\textcolor{Pgruvbox-purple}{9}}}{1}
}
%====================================================================
% 
%====================================================================

% JAVA LSTLISTING STYLE IN Gruvbox Colorscheme
\definecolor{gruvbox-bg}{rgb}{0.282, 0.247, 0.204}
\definecolor{gruvbox-fg1}{rgb}{0.949, 0.898, 0.776}
\definecolor{gruvbox-fg2}{rgb}{0.871, 0.804, 0.671}
\definecolor{gruvbox-red}{rgb}{0.788, 0.255, 0.259}
\definecolor{gruvbox-green}{rgb}{0.518, 0.604, 0.239}
\definecolor{gruvbox-yellow}{rgb}{0.914, 0.808, 0.427}
\definecolor{gruvbox-blue}{rgb}{0.353, 0.510, 0.784}
\definecolor{gruvbox-purple}{rgb}{0.576, 0.412, 0.659}
\definecolor{gruvbox-aqua}{rgb}{0.459, 0.631, 0.737}
\definecolor{gruvbox-gray}{rgb}{0.518, 0.494, 0.471}

\definecolor{lst-bg}{RGB}{45, 45, 45}
\definecolor{lst-fg}{RGB}{220, 220, 204}
\definecolor{lst-keyword}{RGB}{215, 186, 125}
\definecolor{lst-comment}{RGB}{117, 113, 94}
\definecolor{lst-string}{RGB}{163, 190, 140}
\definecolor{lst-number}{RGB}{181, 206, 168}
\definecolor{lst-type}{RGB}{218, 142, 130}


\lstdefinestyle{JavaGruvbox}{
	language=Java,
	basicstyle=\ttfamily\color{lst-fg},
	keywordstyle=\color{lst-keyword},
	keywordstyle=[2]\color{lst-type},
	commentstyle=\itshape\color{lst-comment},
	stringstyle=\color{lst-string},
	numberstyle=\color{lst-number},
	backgroundcolor=\color{lst-bg},
	%frame=tb,
	rulecolor=\color{gruvbox-aqua},
	showstringspaces=false,
	keepspaces=true,
	captionpos=b,
	breaklines=true,
	tabsize=4,
	showspaces=false,
	showtabs=false,
	columns=fullflexible,
	morekeywords={var},
	morekeywords=[2]{boolean, byte, char, double, float, int, long, short, void},
	morecomment=[s]{/}{/},
	morecomment=[l]{//},
	morestring=[b]",
	morestring=[b]',
	numbers=left,
	numbersep=5pt,
	numberstyle=\tiny\color{gray},
}



%====================================================================
% 
%====================================================================


% Define Dracula color scheme for Java
\definecolor{draculawhite-background}{RGB}{237, 239, 252}
\definecolor{draculawhite-comment}{RGB}{98, 114, 164}
\definecolor{draculawhite-keyword}{RGB}{189, 147, 249}
\definecolor{draculawhite-string}{RGB}{152, 195, 121}
\definecolor{draculawhite-number}{RGB}{249, 189, 89}
\definecolor{draculawhite-operator}{RGB}{248, 248, 242}

% Define JavaDraculaWhite lstlisting environment
\lstdefinestyle{JavaDraculaWhite}{
    language=Java,
    backgroundcolor=\color{draculawhite-background},
    commentstyle=\itshape\color{draculawhite-comment},
    keywordstyle=\color{draculawhite-keyword},
    stringstyle=\color{draculawhite-string},
    basicstyle=\ttfamily\footnotesize\color{black},
    identifierstyle=\color{black},
    keywordstyle=\color{draculawhite-keyword}\bfseries,
    morecomment=[s][\color{draculawhite-comment}]{/**}{*/},
    showstringspaces=false,
    showspaces=false,
    breaklines=true,
    frame=single,
    rulecolor=\color{draculawhite-operator},
    tabsize=2,  
	numbers=left,
	numbersep=4pt,
	numberstyle=\ttfamily\tiny\color{gray}
}
%====================================================================
% 
%====================================================================
% Define PythonDraculaWhite lstlisting environment 
\definecolor{draculawhite-bg}{HTML}{FAFAFA}
\definecolor{draculawhite-fg}{HTML}{282A36}
\definecolor{pdraculawhite-keyword}{HTML}{BD93F9}

\definecolor{pdraculawhite-comment}{HTML}{6272A4}
\definecolor{draculawhite-number}{HTML}{FF79C6}


\lstdefinestyle{PythonDraculaWhite}{
    language=Python,
    basicstyle=\ttfamily\small\color{draculawhite-fg},
    backgroundcolor=\color{draculawhite-background},
    keywordstyle=\color{orange}\bfseries,
    stringstyle=\color{draculawhite-string},
    commentstyle=\color{pdraculawhite-comment}\itshape,
    numberstyle=\color{draculawhite-number},
    showstringspaces=false,
	showspaces=false,
    breaklines=true,
	frame=single,
	rulecolor=\color{draculawhite-operator}, 
    tabsize=4,
    morekeywords={as,with,1,2,3,4, 5,6,7,8,9,True,False},
    %escapeinside={(*@}{@*)},
    numbers=left,
    numbersep=5pt,
    %xleftmargin=15pt,
    %framexleftmargin=15pt,
    %framexrightmargin=0pt,
    %framexbottommargin=0pt,
    %framextopmargin=0pt,
    %rulecolor=\color{draculawhite-fg},
    %frame=tb,
    %aboveskip=0pt,
    %belowskip=0pt,
    %captionpos=b,
	numberstyle=\ttfamily\tiny\color{gray} 
}
%====================================================================
% 
%====================================================================

% Define colors for HTML langage
\definecolor{html-orange}{HTML}{FF5733}
\definecolor{html-yellow}{HTML}{F0E130}
\definecolor{html-green}{HTML}{50FA7B}
\definecolor{html-blue}{HTML}{5AFBFF}
\definecolor{html-purple}{HTML}{BD93F9}
\definecolor{html-pink}{HTML}{FF80BF}
\definecolor{html-gray}{HTML}{6272A4}
\definecolor{html-white}{HTML}{F8F8F2}

% Defines a new HTML5 langage that extend on the html langange
\lstdefinestyle{HTMLDraculaWhite}{
  language=HTML,
  backgroundcolor=\color{html-white},
  basicstyle=\ttfamily\color{html-gray},
  keywordstyle=\color{html-blue},
  stringstyle=\color{html-orange},
  commentstyle=\color{html-green},
  tagstyle=\color{html-yellow},
  moredelim=[s][\color{html-pink}]{<!--}{-->},
  moredelim=[s][\color{html-purple}]{\{}{\}},
  showstringspaces=false,
  tabsize=2,
  breaklines=true,
  columns=fullflexible,
  %frame=single,
  framexleftmargin=5mm,
  xleftmargin=10mm,
  numbers=left,
  numberstyle=\tiny\color{html-gray},
  escapeinside={<@}{@>}
}

%====================================================================
% 
%====================================================================
% Define the colors needed for the HTMLDraculaDark environment
\definecolor{htmltag}{HTML}{ff79c6}
\definecolor{htmlattr}{HTML}{f1fa8c}
\definecolor{htmlvalue}{HTML}{bd93f9}
\definecolor{htmlcomment}{HTML}{6272a4}
%\definecolor{htmltext}{HTML}{f8f8f2}
\definecolor{htmltext}{HTML}{401E31}
\definecolor{htmlbackground}{HTML}{282a36}
\definecolor{comphtmlbackground}{HTML}{8093FF}
%\definecolor{htmlbackground}{HTML}{4D5169}

% Define the HTMLDraculaDark environment
\lstdefinestyle{HTMLDraculaDark}{
    basicstyle=\bfseries\ttfamily\color{htmltext},
    commentstyle=\itshape\color{htmlcomment},
    keywordstyle=\bfseries\color{htmltag},
    stringstyle=\color{htmlvalue},
    emph={DOCTYPE,html,head,body,div,span,a,script},
    emphstyle={\color{htmltag}\bfseries},
    sensitive=true,
    showstringspaces=false,
    backgroundcolor=\color{white},
    %frame=tb,
    language=HTML,
    tabsize=4,
    breaklines=true,
    breakatwhitespace=true,
    numbers=left,
    numbersep=10pt,
    numberstyle=\small\bfseries\ttfamily\color{htmlcomment},
    escapeinside={<@}{@>},
	rulecolor=\color{htmlbackground},
	xleftmargin=20pt,
	% Add a vertical line for opening and closing tags
    %frame=lines,
    framesep=2pt,
    framexleftmargin=4pt,
    % Add a vertical line for closing tags that go through multiple lines
    breaklines=true,
    postbreak=\mbox{\textcolor{gray}{$\hookrightarrow$}\space},
    showlines=true,
	% Add a rule to apply commentstyle to keywords inside comments
    moredelim=[s][\itshape\color{htmlcomment}]{<!--}{-->},
    morekeywords={id,class,type,name,value,placeholder,checked,src,href,alt}
}




%====================================================================
% 
%====================================================================






% Crée un environnement "Theorem" numéroté en fonction du document
\tcbuselibrary{theorems,skins,hooks} 
\newtcbtheorem{Theorem}{Théorème}
{%
	enhanced,
	breakable,
	colback = mytheorembg,
	frame hidden,
	boxrule = 0sp,
	borderline west = {2pt}{0pt}{mytheoremfr},
	sharp corners,
	detach title,
	before upper = \tcbtitle\par\smallskip,
	coltitle = mytheoremfr,
	fonttitle = \bfseries\fontfamily{lmss}\selectfont,
	description font = \mdseries\fontfamily{lmss}\selectfont,
	separator sign none,
	segmentation style={solid, mytheoremfr},
}
{thm}

% Crée un environnement "Preuve" numéroté en fonction du document
\tcbuselibrary{theorems,skins,hooks}
\newtcbtheorem{Preuve}{Preuve}
{
	enhanced,
	breakable,
	colback=white,
	frame hidden,
	boxrule = 0sp,
	borderline west = {2pt}{0pt}{mytheoremfr},
	sharp corners,
	detach title,
	before upper = \tcbtitle\par\smallskip,
	coltitle = mytheoremfr,
	description font=\fontfamily{lmss}\selectfont,
	fonttitle=\fontfamily{lmss}\selectfont\bfseries,
	separator sign none,
	segmentation style={solid, mytheoremfr},
}
{th}


% Crée un environnement "Preuve" numéroté en fonction du document
\tcbuselibrary{theorems,skins,hooks}
\newtcbtheorem{Explication}{Explication}
{
	enhanced,
	breakable,
	colback=white,
	frame hidden,
	boxrule = 0sp,
	borderline west = {2pt}{0pt}{mytheoremfr},
	sharp corners,
	detach title,
	before upper = \tcbtitle\par\smallskip,
	coltitle = mytheoremfr,
	description font=\fontfamily{lmss}\selectfont,
	fonttitle=\fontfamily{lmss}\selectfont\bfseries,
	separator sign none,
	segmentation style={solid, mytheoremfr},
}
{th}




% Crée un environnement "Example" numéroté en fonction du document
\tcbuselibrary{theorems,skins,hooks}
\newtcbtheorem{Example}{Exemple.}
{
	enhanced,
	breakable,
	colback=lightBlue,
	frame hidden,
	boxrule = 0sp,
	borderline west = {2pt}{0pt}{myb},
	sharp corners,
	detach title,
	before upper = \tcbtitle\par\smallskip,
	coltitle = myb,
	description font=\fontfamily{lmss}\selectfont,
	fonttitle=\fontfamily{lmss}\selectfont\bfseries,
	separator sign none,
	segmentation style={solid, mytheoremfr},
}
{th}



% Crée un environnement "EExample" numéroté en fonction du document
\tcbuselibrary{theorems,skins,hooks}
\newtcbtheorem{EExample}{Exemple}
{
	enhanced,
	breakable,
	colback=white,
	frame hidden,
	boxrule = 0sp,
	borderline west = {2pt}{0pt}{myb},
	sharp corners,
	detach title,
	before upper = \tcbtitle\par\smallskip,
	coltitle = myb,
	description font=\mdseries\fontfamily{lmss}\selectfont,
	fonttitle=\fontfamily{lmss}\selectfont\bfseries,
	separator sign none,
	segmentation style={solid, mytheoremfr},
}
{th}



% Crée un environnement "Lemme" numéroté en fonction du document
\tcbuselibrary{theorems,skins,hooks}
\newtcbtheorem{Lemme}{Lemme}
{
	enhanced,
	breakable,
	colback=mylenmabg,
	frame hidden,
	boxrule = 0sp,
	borderline west = {2pt}{0pt}{mylenmafr},
	sharp corners,
	detach title,
	before upper = \tcbtitle\par\smallskip,
	coltitle = mylenmafr,
	description font=\mdseries\fontfamily{lmss}\selectfont,
	fonttitle=\fontfamily{lmss}\selectfont\bfseries,
	separator sign none,
	segmentation style={solid, mytheoremfr},
}
{th}


\tcbuselibrary{theorems,skins,hooks}
\newtcbtheorem{PreuveL}{Preuve.}
{
	enhanced,
	breakable,
	colback=white,
	frame hidden,
	boxrule = 0sp,
	borderline west = {2pt}{0pt}{mylenmafr},
	sharp corners,
	detach title,
	before upper = \tcbtitle\par\smallskip,
	coltitle = mylenmafr,
	description font=\fontfamily{lmss}\selectfont,
	fonttitle=\fontfamily{lmss}\selectfont\bfseries,
	separator sign none,
	segmentation style={solid, mytheoremfr},
}
{th}


\newtcbtheorem{Remarque}{Remarque}
{
	enhanced,
	breakable,
	colback=white,
	frame hidden,
	boxrule = 0sp,
	borderline west = {2pt}{0pt}{myb},
	sharp corners,
	detach title,
	before upper = \tcbtitle\par\smallskip,
	coltitle = myb,
	description font=\mdseries\fontfamily{lmss}\selectfont,
	fonttitle=\fontfamily{lmss}\selectfont\bfseries,
	separator sign none,
	segmentation style={solid, mytheoremfr},
}
{th}


\newtcbtheorem{DefG}{Définition}
{
	enhanced,
	breakable,
	colback=mygbg,
	frame hidden,
	boxrule = 0sp,
	borderline west = {2pt}{0pt}{myg},
	sharp corners,
	detach title,
	before upper = \tcbtitle\par\smallskip,
	coltitle = myg,
	description font=\mdseries\fontfamily{lmss}\selectfont,
	fonttitle=\fontfamily{lmss}\selectfont\bfseries,
	separator sign none,
	segmentation style={solid, mytheoremfr},
}
{th}



% Crée une boîte ayant la même couleur que l'environnement theorem.
\tcbuselibrary{theorems,skins,hooks}
\newtcolorbox{Theoremcon}
{%
	enhanced
	,breakable
	,colback = mytheorembg
	,frame hidden
	,boxrule = 0sp
	,borderline west = {2pt}{0pt}{mytheoremfr}
	,sharp corners
	,description font = \mdseries
	,separator sign none
}

% Crée un environnement "Definition" numéroté en fonction de la section
\newtcbtheorem[number within=chapter]{Definition}{Définition}{enhanced,
	before skip=2mm,after skip=2mm, colback=red!5,colframe=red!80!black,boxrule=0.5mm,
	attach boxed title to top left={xshift=1cm,yshift*=1mm-\tcboxedtitleheight}, varwidth boxed title*=-3cm,
	boxed title style={frame code={
			\path[fill=tcbcolback!10!red]
			([yshift=-1mm,xshift=-1mm]frame.north west)
			arc[start angle=0,end angle=180,radius=1mm]
			([yshift=-1mm,xshift=1mm]frame.north east)
			arc[start angle=180,end angle=0,radius=1mm];
			\path[left color=tcbcolback!10!myred,right color=tcbcolback!10!myred,
			middle color=tcbcolback!60!myred]
			([xshift=-2mm]frame.north west) -- ([xshift=2mm]frame.north east)
			[rounded corners=1mm]-- ([xshift=1mm,yshift=-1mm]frame.north east)
			-- (frame.south east) -- (frame.south west)
			-- ([xshift=-1mm,yshift=-1mm]frame.north west)
			[sharp corners]-- cycle;
		},interior engine=empty,
	},
	fonttitle=\bfseries,
	title={#2},#1}{def}

% Crée un environnement "definition" numéroté en fonction du Chapitre
\newtcbtheorem[number within=section]{definition}{Définition}{enhanced,
	before skip=2mm,after skip=2mm, colback=red!5,colframe=red!80!black,boxrule=0.5mm,
	attach boxed title to top left={xshift=1cm,yshift*=1mm-\tcboxedtitleheight}, varwidth boxed title*=-3cm,
	boxed title style={frame code={
			\path[fill=tcbcolback]
			([yshift=-1mm,xshift=-1mm]frame.north west)
			arc[start angle=0,end angle=180,radius=1mm]
			([yshift=-1mm,xshift=1mm]frame.north east)
			arc[start angle=180,end angle=0,radius=1mm];
			\path[left color=tcbcolback!60!black,right color=tcbcolback!60!black,
			middle color=tcbcolback!80!black]
			([xshift=-2mm]frame.north west) -- ([xshift=2mm]frame.north east)
			[rounded corners=1mm]-- ([xshift=1mm,yshift=-1mm]frame.north east)
			-- (frame.south east) -- (frame.south west)
			-- ([xshift=-1mm,yshift=-1mm]frame.north west)
			[sharp corners]-- cycle;
		},interior engine=empty,
	},
	fonttitle=\bfseries,
	title={#2},#1}{def}

\usetikzlibrary{arrows,calc,shadows.blur}
\tcbuselibrary{skins}
\newtcolorbox{note}[1][]{%
	enhanced jigsaw,
	colback=gray!20!white,%
	colframe=gray!80!black,
	size=small,
	boxrule=1pt,
	title=\textbf{Note : },
	halign title=flush center,
	coltitle=black,
	breakable,
	drop shadow=black!50!white,
	attach boxed title to top left={xshift=1cm,yshift=-\tcboxedtitleheight/2,yshifttext=-\tcboxedtitleheight/2},
	minipage boxed title=1.5cm,
	boxed title style={%
		colback=white,
		size=fbox,
		boxrule=1pt,
		boxsep=2pt,
		underlay={%
			\coordinate (dotA) at ($(interior.west) + (-0.5pt,0)$);
			\coordinate (dotB) at ($(interior.east) + (0.5pt,0)$);
			\begin{scope}
				\clip (interior.north west) rectangle ([xshift=3ex]interior.east);
				\filldraw [white, blur shadow={shadow opacity=60, shadow yshift=-.75ex}, rounded corners=2pt] (interior.north west) rectangle (interior.south east);
			\end{scope}
			\begin{scope}[gray!80!black]
				\fill (dotA) circle (2pt);
				\fill (dotB) circle (2pt);
			\end{scope}
		},
	},
	#1,
}


% Crée un environnement "qstion" 
\newtcbtheorem{qstion}{Question}{enhanced,
	breakable,
	colback=white,
	colframe=mygr,
	attach boxed title to top left={yshift*=-\tcboxedtitleheight},
	fonttitle=\bfseries,
	title={#2},
	boxed title size=title,
	boxed title style={%
		sharp corners,
		rounded corners=northwest,
		colback=tcbcolframe,
		boxrule=0pt,
	},
}{def}


% Pour créer un environnement "Liste" 

\tcbuselibrary{theorems,skins,hooks}
\newtcbtheorem[number within=section]{Liste}{Liste}
{%
	enhanced
	,breakable
	,colback = myp!10
	,frame hidden
	,boxrule = 0sp
	,borderline west = {2pt}{0pt}{myp!85!black}
	,sharp corners
	,detach title
	,before upper = \tcbtitle\par\smallskip
	,coltitle = myp!85!black
	,fonttitle = \bfseries\sffamily
	,description font = \mdseries
	,separator sign none
	,segmentation style={solid, myp!85!black}
}
{th}


\tcbuselibrary{theorems,skins,hooks}
\newtcbtheorem{Syntaxe}{Syntaxe.}
{%
	enhanced
	,breakable
	,colback = myp!10
	,frame hidden
	,boxrule = 0sp
	,borderline west = {2pt}{0pt}{myp!85!black}
	,sharp corners
	,detach title
	,before upper = \tcbtitle\par\smallskip
	,coltitle = myp!85!black
	,fonttitle = \bfseries\fontfamily{lmss}\selectfont 
	,description font = \mdseries\fontfamily{lmss}\selectfont 
	,separator sign none
	,segmentation style={solid, myp!85!black}
}
{th}



% Crée un environnement "Concept" numéroté en fonction du document
\tcbuselibrary{theorems,skins,hooks}
\newtcbtheorem{Concept}{Concept.}
{
	enhanced,
	breakable,
	colback=mylenmabg,
	frame hidden,
	boxrule = 0sp,
	borderline west = {2pt}{0pt}{mylenmafr},
	sharp corners,
	detach title,
	before upper = \tcbtitle\par\smallskip,
	coltitle = mylenmafr,
	description font=\mdseries\fontfamily{lmss}\selectfont,
	fonttitle=\fontfamily{lmss}\selectfont\bfseries,
	separator sign none,
	segmentation style={solid, mytheoremfr},
}
{th}


% Crée un environnement "codeEx" numéroté en fonction du document
\tcbuselibrary{theorems,skins,hooks}
\newtcbtheorem{codeEx}{Exemple}
{
	enhanced,
	breakable,
	colback=white,
	frame hidden,
	boxrule = 0sp,
	borderline west = {2pt}{0pt}{gruvbox-bg},
	sharp corners,
	detach title,
	before upper = \tcbtitle\par\smallskip,
	coltitle = gruvbox-bg,
	description font=\md:wqseries\fontfamily{lmss}\selectfont,
	fonttitle=\fontfamily{lmss}\selectfont\bfseries,
	separator sign none,
	segmentation style={solid, mytheoremfr},
}
{th}


% Crée un environnement "codeEx" numéroté en fonction du document
\tcbuselibrary{theorems,skins,hooks}
\newtcbtheorem{codeRem}{Remarque.}
{
	enhanced,
	breakable,
	colback=white,
	frame hidden,
	boxrule = 0sp,
	borderline west = {2pt}{0pt}{gruvbox-bg},
	sharp corners,
	detach title,
	before upper = \tcbtitle\par\smallskip,
	coltitle = gruvbox-bg,
	description font=\mdseries\fontfamily{lmss}\selectfont,
	fonttitle=\fontfamily{lmss}\selectfont\bfseries,
	separator sign none,
	segmentation style={solid, mytheoremfr},
}
{th}


\tcbuselibrary{theorems,skins,hooks}
\newtcbtheorem{Identite}{Identité.}
{
	enhanced,
	breakable,
	colback=white,
  before upper=\tcbtitle\par\Hugeskip,
	frame hidden,
	boxrule = 0sp,
	borderline west = {2pt}{0pt}{gruvbox-bg},
	sharp corners,
	detach title,
	before upper = \tcbtitle\par\smallskip,
	coltitle = gruvbox-bg,
	description font=\mdseries\fontfamily{lmss}\selectfont,
	fonttitle=\fontfamily{lmss}\selectfont\bfseries,
	fontlower=\fontfamily{cmr}\selectfont,
  separator sign none,
	segmentation style={solid, mytheoremfr},
}
{th}

\tcbuselibrary{theorems,skins,hooks}
\newtcbtheorem{Exercice}{Exercice}
{
	enhanced,
	breakable,
	colback=white,
  before upper=\tcbtitle\par\Hugeskip,
	frame hidden,
	boxrule = 0sp,
	borderline west = {2pt}{0pt}{gruvbox-green},
	sharp corners,
	detach title,
	before upper = \tcbtitle\par\smallskip,
	coltitle = gruvbox-green,
	description font=\mdseries\fontfamily{lmss}\selectfont,
	fonttitle=\fontfamily{lmss}\selectfont\bfseries,
	fontlower=\fontfamily{cmr}\selectfont,
  separator sign none,
	segmentation style={solid, mytheoremfr},
}
{th}

% Crée un environnement "Réponse" numéroté en fonction du document
\tcbuselibrary{theorems,skins,hooks}
\newtcbtheorem{Reponse}{Réponse}
{
	enhanced,
	breakable,
	colback=white,
	frame hidden,
	boxrule = 0sp,
	borderline west = {2pt}{0pt}{mytheoremfr},
	sharp corners,
	detach title,
	before upper = \tcbtitle\par\smallskip,
	coltitle = mytheoremfr,
	description font=\fontfamily{lmss}\selectfont,
	fonttitle=\fontfamily{lmss}\selectfont\bfseries,
	separator sign none,
	segmentation style={solid, mytheoremfr},
}
{th}

\newtcbtheorem{Definitionx}{Définition}
{
enhanced,
breakable,
colback=red!5,
  before upper=\tcbtitle\par\Hugeskip,
frame hidden,
boxrule = 0sp,
borderline west = {2pt}{0pt}{red!80!black},
sharp corners,
detach title,
before upper = \tcbtitle\par\smallskip,
coltitle = red!80!black,
description font=\mdseries\fontfamily{lmss}\selectfont,
fonttitle=\fontfamily{lmss}\selectfont\bfseries,
fontlower=\fontfamily{cmr}\selectfont,
  separator sign none,
segmentation style={solid, mytheoremfr},
}
{th}

\tcbuselibrary{theorems,skins,hooks}
\newtcbtheorem[number within=chapter]{prop}{Proposition}
{%
	enhanced,
	breakable,
	colback = mypropbg,
	frame hidden,
	boxrule = 0sp,
	borderline west = {2pt}{0pt}{mypropfr},
	sharp corners,
	detach title,
	before upper = \tcbtitle\par\smallskip,
	coltitle = mypropfr,
	fonttitle = \bfseries\sffamily,
	description font = \mdseries,
	separator sign none,
	segmentation style={solid, mypropfr},
}
{th}


\tcbuselibrary{theorems,skins,hooks}
\newtcbtheorem[number within=section]{Prop}{Proposition}
{%
	enhanced,
	breakable,
	colback = mypropbg,
	frame hidden,
	boxrule = 0sp,
	borderline west = {2pt}{0pt}{mypropfr},
	sharp corners,
	detach title,
	before upper = \tcbtitle\par\smallskip,
	coltitle = mypropfr,
	fonttitle = \bfseries\sffamily,
	description font = \mdseries,
	separator sign none,
	segmentation style={solid, mypropfr},
}
{th}


%================================
% Corollery
%================================
\tcbuselibrary{theorems,skins,hooks}
\newtcbtheorem[number within=section]{Corollary}{Corollary}
{%
	enhanced
	,breakable
	,colback = myp!10
	,frame hidden
	,boxrule = 0sp
	,borderline west = {2pt}{0pt}{myp!85!black}
	,sharp corners
	,detach title
	,before upper = \tcbtitle\par\smallskip
	,coltitle = myp!85!black
	,fonttitle = \bfseries\sffamily
	,description font = \mdseries
	,separator sign none
	,segmentation style={solid, myp!85!black}
}
{th}
\tcbuselibrary{theorems,skins,hooks}
\newtcbtheorem[number within=chapter]{corollary}{Corollaire}
{%
	enhanced
	,breakable
	,colback = myp!10
	,frame hidden
	,boxrule = 0sp
	,borderline west = {2pt}{0pt}{myp!85!black}
	,sharp corners
	,detach title
	,before upper = \tcbtitle\par\smallskip
	,coltitle = myp!85!black
	,fonttitle = \bfseries\sffamily
	,description font = \mdseries
	,separator sign none
	,segmentation style={solid, myp!85!black}
}
{th}




\usepackage{amsmath,amsthm,mathtools}
\usepackage{titlesec}
\usepackage{microtype}
\definecolor{myblue}{RGB}{0,82,155}



\usepackage[scr]{rsfso}
%\usepackage{libertine}
%\usepackage[euler-digits]{eulervm}
%\usepackage{mathpazo}
%usepackage{palatino}
%usepackage{crimson}


\DeclareRobustCommand{\looongrightarrow}{%
  \DOTSB\relbar\joinrel\relbar\joinrel\relbar\joinrel\rightarrow
}


\title{\Huge{Structures de Données}\\{IFT2015}\\{\textbf{Devoir 1}}}
\author{\huge{Franz Girardin}}
\date{\today}
\lstset{inputencoding=utf8/latin1}






            %%%%%%%%%%%%%%%%%  Sect.                          %%%%%%%%%%%%%%%%%%%%%%%%%%%%%%%%%%%%%%%%%%%%%%%%%%%%%%%%%
\titleformat{\chapter}[display]
  {\normalfont\bfseries\color{black}}
  {\filleft%
    \begin{tikzpicture}
    \node[
      outer sep=0pt,
      text width=1.5cm,
      minimum height=2cm,
      fill=draculawhite-background,
      font=\color{black}\fontsize{40}{50}\selectfont,
      align=center
      ] (num) {\thechapter};
    \node[
      rotate=90,
      anchor=south,
      font=\color{black}\Large\normalfont
      ] at ([xshift=-5pt]num.west) {\textls[180]{\textsc{\textit{Section}}}};  
    \end{tikzpicture}%
  }
  {5pt}
  {\titlerule[2.0pt]\vskip3pt\titlerule\vskip4pt\large\normalfont}



\titleformat{\section}
  {\normalfont\scshape}{\thesection}{1em}{}


% Customizing the spacing for the chapter titles
\titlespacing*{\chapter}{0pt}{0pt}{20pt}

% Allow hfill in math environment
\newcommand{\specialcell}[1]{\ifmeasuring@#1\else\omit$\displaystyle#1$\ignorespaces\fi}

% Allow you to do the non implication (implication barred)
\newcommand{\notimplies}{%
  \mathrel{{\ooalign{\hidewidth$\not\phantom{=}$\hidewidth\cr$\implies$}}}}



\DeclareRobustCommand{\looongrightarrow}{%
  \DOTSB\relbar\joinrel\relbar\joinrel\relbar\joinrel\rightarrow
}


\DeclareMathOperator{\di}{d\!}
\newcommand*\Eval[3]{\left.#1\right\rvert_{#2}^{#3}}

\begin{document}
\begin{titlepage} % Suppresses displaying the page number on the title page and the subsequent page counts as page 1
	\newcommand{\HRule}{\rule{\linewidth}{0.5mm}} % Defines a new command for horizontal lines, change thickness here
	
	\center % Centre everything on the page
	
	%------------------------------------------------
	%	Headings
	%------------------------------------------------
	
	\textsc{\LARGE Université de Montréal}\\[1.5cm] % Main heading such as the name of your university/college
	
	\textsc{\Large Département d'Informatique et de Recherche Opérationnelle}\\[0.5cm] % Major heading such as course name
	
	\textsc{\large Structures de données}\\[0.5cm] % Minor heading such as course title
	
	%------------------------------------------------
	%	Title
	%------------------------------------------------
	
	\HRule\\[0.4cm]
	
	{\huge\bfseries Analyse de Complexité et Piles}\\[0.4cm] % Title of your document
	
	\HRule\\[1.5cm]
	
	%------------------------------------------------
	%	Author(s)
	%------------------------------------------------
	
	\begin{minipage}{0.4\textwidth}
		\begin{flushleft}
			\large
			\textit{Auteur}\\
		Franz \textsc{Girardin} % Your name
		\end{flushleft}
	\end{minipage}
	~
	\begin{minipage}{0.4\textwidth}
		\begin{flushright}
			\large
			\textit{Matricule}\\
			\textsc{20078678} % Supervisor's name
		\end{flushright}
	\end{minipage}
	
	% If you don't want a supervisor, uncomment the two lines below and comment the code above
	%{\large\textit{Author}}\\
	%John \textsc{Smith} % Your name
	
	%------------------------------------------------
	%	Date
	%------------------------------------------------
	
	\vfill\vfill\vfill % Position the date 3/4 down the remaining page
	
	{\large\today} % Date, change the \today to a set date if you want to be precise
	
	%------------------------------------------------
	%	Logo
	%------------------------------------------------
	
	%\vfill\vfill
	%\includegraphics[width=0.2\textwidth]{placeholder.jpg}\\[1cm] % Include a department/university logo - this will require the graphicx package
	 
	%----------------------------------------------------------------------------------------
	
	\vfill % Push the date up 1/4 of the remaining page
	
\end{titlepage}
\pagebreak
\tableofcontents
\pagebreak


  \vspace{-2em}
  \chapter{Analyse de complexité}

  \vspace{-2em}
  \begin{Exercice}{}{}
    Compte tenu des fonctions mystérieuses suivantes, pour chacune d’elles, 
    déterminer quelle est la complexité dans le temps et l’espace de son exécution 
    (Big O) et expliquer ce que vous pensez que la fonction fait. Les réponses 
    simples ne seront pas acceptées, il est nécessaire de justifier votre réponse.
    Par exemple : Si dans l’exercice la récursion est utilisée, vous pouvez 
    soutenir votre justification en présentant l’arbre de récursion. 
    Toutes vos réponses doivent être incluses dans votre rapport.
  \end{Exercice}
  \section{\texttt{MistFonction1}}
  \vspace{-3em}
  \vspace{-1em}

  \begin{lstlisting}[style=JavaDraculaWhite]
public class MistFonction1{

  public static int mistFonction1(int m, int n) {
      if (m == 1) && (n == 1) return 1; 
      if (m == 0) || (n == 0) return 0
      return mistFonction1(m -1, n) + mistFonction1(m, n -1)
  }
}\end{lstlisting}
  \vspace{-2em}
  \paragraph{}
  La fonction présente deux \textbf{cas de bases}, dont l'un retourne 
  la valeur \texttt{1} lorsque \textit{les deux} arguments ont 
  pour valeur $1$, alors que l'autre retourne $0$ sous la condition
  que \textit{l'un des deux} arguments a pour valur $0$. 
  \vspace{-1em}
  \paragraph{}
  La seconde portion de la fonction engendre des appels récursifs 
  qui prennent fin lorsque les cas de bases sont rencontré. Nous constatons 
  alors que chaque appel de fonction \textit{hors base} engendre 
  \textbf{deux appels}, jusqu'à ce que les conditions d'arrêts soient effectives. 
  
  \begin{figure}[H]
 \dirtree{%
.1 f(3, 3) $\looongrightarrow$ \textcolor{blue}{3} + \textcolor{blue}{3} =    \textcolor{red}{6}.
.2 f(2, 3) $\looongrightarrow$ \textcolor{blue}{3}.
.3 f(1, 3)  $\looongrightarrow$ \textcolor{myg}{1}.
.4 f(0, 3) $\looongrightarrow$ 0 \textbf{cas de base}.
.4 f(1, 2) $\looongrightarrow$ \textcolor{myg}{1}  .
.5 f(0, 2) $\looongrightarrow$ 0 \textbf{base}.
.5 f(1, 1) $\looongrightarrow$ 1 \textbf{base}.
.3 f(2, 2) $\looongrightarrow$ \textcolor{myg}{2}.
.4 f(1, 2) $\looongrightarrow$ \textcolor{myg}{1} (répète le sous-arbre ci-dessus).
.4 f(2, 1) $\looongrightarrow$ \textcolor{myg}{1}.
.5 f(1, 1) $\looongrightarrow$ 1 \textbf{base}.
.5 f(2, 0) $\looongrightarrow$ 0 \textit{cas de base}.
.2 f(3, 2) $\looongrightarrow$ \textcolor{blue}{3} .
.3 f(2, 2) $\looongrightarrow$ \textcolor{myg}{2} (répète le sous-arbre ci-dessus).
.4 f(1, 2) répète. 
.4 f(2, 1) répète. 
.3 f(3, 1) $\looongrightarrow$ \textcolor{myg}{1}.
.4 f(2, 1) répète. 
.4 f(3, 0)  $\looongrightarrow$ 0 \textbf{base}  .
}
\caption{Exemple de $f(3,3)$}
  \end{figure}
  \paragraph{}
  Ainsi, nous pouvons représenter 
  les appels récursifs par une arbre tel qu'à chaque 
  niveau de l'arbre, \textbf{le nombre d'appels est doublé} 
  par rapport au précédent.
  En considérant $j$ comme étant la valeur de $m$ lorsque $m$ atteint $0$, 
  et $k$, la valeur de $n$ lorsque $n$ atteint $1$, on obtient 
  l'arbre généralisé à tous les cas possibles qui est présenté à la figure 1.2. 



  \vspace{-1em}
  \begin{figure}[H]
     
    \dirtree{%
  .1 f(m, n).
  .2 f(m-1, n).
  .3 f(m-2, n).
  .4 $\vdots$.
  .5 f(j+1, n).
  .6 f(j, n) $\looongrightarrow$ 0 \textbf{base}.
  .6 f(j+1, n-1).
  .7 $\vdots$.
  .8 f(j+1, k+1).
  .9 f(j, k+1) $\looongrightarrow$ 0 \textbf{base}.
  .9 f(j+1, k) $\looongrightarrow$  \textbf{base}.
  .8 f(j+1, k) $\looongrightarrow$  \textbf{base}.
  .3 f(m-1, n-1).
  .4 $\vdots$.
  .2 f(m, n-1).
  .3 f(m-1, n-1).
  .4 $\vdots$.
  .5 f(j+1, n-1).
  .6 f(j, n-1) $\looongrightarrow$ 0 \textbf{base}.
  .6 f(j+1, n-2).
  .7 $\vdots$.
  .8 f(j+1, k+1).
  .9 f(j, k+1) $\looongrightarrow$ 0 \textbf{base}.
  .9 f(j+1, k) $\looongrightarrow$  \textbf{base}.
  .8 f(j+1, k) $\looongrightarrow$  \textbf{base}.
  .3 f(m, n-2).
  .4 $\vdots$.
  }
  \caption{Stucture arboressante de la fonction}
  \end{figure}

  \paragraph{Complexité dans le temps}
  \mbox{}\\
  La complexité temporelle de cette fonction est \textit{exponentielle}, puisqu'à 
  chaque appel récursif, deux nouveaux appels sont engendrés. Par ailleurs, 
  chaque argument $m$ et $n$ peut augmenter la complexité. Nous avons donc : 
  \[ O( m +n )^2 \]

  \begin{note}{}{}
      L'arbre est \textbf{symétrique} : 
      \[ \forall n, m \in \mathbb{N}, f(m, n) = f(n, m) \]   
      Nous avons d'ailleur montré que $f(3, 2) = f(2, 3)$
  \end{note}        
  \begin{figure*}
   \begin{lstlisting}[style=JavaDraculaWhite]
 public class MistFunction2 {
    public static List<List<String>> mistFunction2(String target, List<String> pieces) {
        List<List<String>>[] table = new ArrayList[target.length() + 1];
        for (int i = 0; i <= target.length(); i++) {
            table[i] = new ArrayList<>();
        }
        table[0].add(new ArrayList<>());

        for (int i = 0; i < target.length(); i++) {
            for (String piece : pieces) {
                if (i + piece.length() <= target.length() &&
                    target.startsWith(piece, i)) {
                    List<List<String>> newCombinations = new ArrayList<>();
                    for (List<String> subarray : table[i]) {
                        List<String> newSubarray = new ArrayList<>(subarray);
                        newSubarray.add(piece);
                        newCombinations.add(newSubarray);
                    }
                    table[i + piece.length()].addAll(newCombinations);
                }
            }
        }

        return table[target.length()];
    }
}
    
  \end{lstlisting}    
  \end{figure*}
  \paragraph{Complexité dans l'espace}
  \mbox{}\\
  Chaque appel résursif est additionné dans 
  \textbf{la pile d'appels} qui emmagasine en mémoire 
  quelles fonctions parents sont appelées par les fonction enfants. 
  La complexité dans l'espace correspond donc à la hateur de la pile d'appel 
  qui est elle-même liée à la profondeur maximale de l'arbre. 


  
  La profondeur maximale de l'arbre correspond 
  au chemin le plus long de la racine de l'arbre d'appels (lorsque f(m, n) 
  est appelé pour la première fois) jusqu'à une feuille de l'arbre 
  (lorsque une condition de base est atteinte). 
  Dans ce cas, la profondeur maximale est atteinte lorsqu'on 
  suit toujours le chemin $f(m - 1, n)  $ou $f(m, n - 1)$ jusqu'à
  ce que m ou n atteigne 0. Si $m \neq n$, l'une des deux direction sera plus longue. 
  Il faut donc considérer $\max(m, n)$. 

  On peut alors conclure que la complexité
  en espace de cet algorithme est : 
  \[ O(\max(m, n)) \]


  \paragraph{Raison dêtre}
  L'algorithme semble compter le nombre de façons possibles de se déplacer 
  dans une grille $m \times n$ avec comme contrainte de se déplacer uniquement 
  vers le bas ($m$) ou uniquement vers la drotie $(n)$
  \section{\texttt{MistFonction2}} 
  \paragraph{Raison d'être}
  Soit une chaine de caractères \texttt{target} et une liste 
  \texttt{pieces} dont chaque élément est un caractère de  
  la chaîne \texttt{target}, la fonction engendre une 
  liste de listes de chaînes de charactère. Cette liste correspond 
  à \textit{toutes les combinaisons possibles} de caractères permmetant 
  d'obtenir la chaîne originale. 

  \begin{EExample}{}{}
    Soit \texttt{target = "abc"} et \texttt{piece = ["a", "b", "c"]}, on obtient 
    alors le résultat suivant : 
    \begin{itemize}
      \item [$\rhd$] \texttt{table = } \textcolor{red}{[}
        \\ \texttt{\textcolor{myb}{[}[]\textcolor{myb}{]}}, 
        \texttt{\textcolor{myb}{[}["a"]\textcolor{myb}{]}}, 
        \texttt{\textcolor{myb}{[}["ab"], ["a", "b"]\textcolor{myb}{]}}, \\
        \texttt{\textcolor{myb}{[}["a", "bc"], ["ab", "c"]}, \\ 
        \texttt{["a", "b", "c"]\textcolor{myb}{]}} \\
        \textcolor{red}{]}
      \item [$\rhd$] \texttt{table[target.length()] =}  
        \texttt{\textcolor{myb}{[}["a", "bc"], ["ab", "c"]},  
        \texttt{["a", "b", "c"]\textcolor{myb}{]}} \\
        \textcolor{red}{]}
      \item[$\blacktriangleright$] $f($\texttt{target, pieces}$)$ \texttt{=} 
        \texttt{\textcolor{myb}{[}["a", "bc"], ["ab", "c"]},  
        \texttt{["a", "b", "c"]\textcolor{myb}{]}} \\
        \textcolor{red}{]}


 
    \end{itemize}
     
  \end{EExample}


  \paragraph{Complexité dans le temps}
  La fonctions possède deux boucles imbriquées de complexitée proportionnelle à
  \texttt{target.lenght()}. En effet,on suppose que \texttt{pieces} est une partitions de \texttt{target}   
  tel que :
  \[ \texttt{target.lenght() = pieces.lenght}  \]
  Dans la seconde boucle, on vérifie si une pièce, disons de longueur $k$, 
  est le début d'une sous-chaîne. En supposant que $k$ est de longueur 1, cette opération 
  est à temps constant. Autrement, elle dépendrait de la longueur de $k$.


  Finalement, pour chaque combinaison existante à la position $i$ dans \texttt{table}, 
  la fonction tente de l'étendre en ajoutant une nouvelle \texttt{piece} valide. 
  Cette opération de génération de combinaison est dépendante du nombre de
  combinaisons précédentes $C_i$, qui peuvent croître de manière significative 
  à chaque étape. Ainsi, l'opération de génération de combinaison a une 
  complexité temporelle de $O(n \cdot C)$, où $n$ est la longueur de la 
  cible et $C$ est le nombre total de combinaisons uniques à l'indice $i$. 
  Si chaque \texttt{piece} peut être utilisée une seule fois et correspond 
  à un caractère unique dans \texttt{target}, le nombre de combinaisons $C_i$ 
  à chaque indice reste constant, simplifiant la complexité globale de la fonction à $O(n^2)$.
  Cependant, dans des cas où des caractères se répètent et peuvent être combinés de
  différentes manières, $C$ peut croître exponentiellement, augmentant ainsi l
  a complexité temporelle.


  \paragraph{Complexité dans l'espace}
  La complexité en espace totale est la somme de la mémoire requise pour le tableau 
  \texttt{table} et les combinaisons qu'il contient. Dans le pire cas, avec un grand nombre de 
  combinaisons, elle peut devenir exponentielle, notée \( O(2^n) \). Toutefois, si chaque 
  \texttt{piece} est unique et ne peut être utilisée qu'une seule fois, la complexité est 
  réduite à \( O(n^2) \), car chaque indice contiendrait au plus une combinaison de \texttt{n} 
  caractères, et il y a \texttt{n+1} indices.

  \section{\texttt{MistFonction3}}
  \begin{lstlisting}[style=JavaDraculaWhite]
 public class MistFunction3 {
    public static boolean mistFunction3(int target, int[] options) {
        boolean[] table = new boolean[target + 1];
        table[0] = true;

        for (int i = 0; i <= target; i++) {
            if (table[i]) {
                for (int option : options) {
                    if (i + option <= target) {
                        table[i + option] = true;
                    }
                }
            }
        }
        return table[target];
    }
}    
  \end{lstlisting}


  \paragraph{Raison d'être}
  La fonction vérifie s'il est possible d'additionner une combinaison 
  d'entiers présents 
  dans le tableau d'options \texttt{options} afin d'obtenir le nombre 
  \texttt{target}.


  \paragraph{Complexité dans le temps}
  La complexité temporelle est déterminée par les deux boucles imbriquées 
  itératives. La boucle extérireure s'exécute une seule fois de 
  \texttt{0} à \texttt{target}. Plus l'entier \texttt{target} est grand, 
  plus grand sera le nombre d'itérations. Pour cette portion, nous obtenons alors 
  un complexité de $O(n)$ où $n$ est la valur de \texttt{target}. Par ailleurs, 
  pour chaque itération de la boucle extérireure, nous effectuons $m$ itérations 
  de la boucles intérieure où $m$ est le nombre de possibilité dans le 
  tableau \texttt{options}. Nous avons donc une complexité :
           \[ O( n \cdot m)\]
  Or, il est possible que le nombre d'options fournies soit considérablement 
  inférieure à la valeur de \texttt{target}. Dans ce cas, la complexité         
  sera quasi linéaire.  
          \[O(n \cdot m) \stackrel{m \to 1}{\looongrightarrow} O(n) \]


  \paragraph{Complexité dans l'espace}
  L'opération la plus déterminante pour la complexité spaciale est la création 
  du tableau \texttt{table}. Ce tableau contient une entrée de plus que la taille de l'entier 
  \texttt{target}. Ainsi, plus l'entier \texttt{target} est grand, plus grand 
  sera le tableau \texttt{table} et plus grande sera la consommation de mémoire. 
  La complexité spaciale évolue donc linéairement avec $n$ où $n$ est la 
  valeur de \texttt{target} :
  \[ O(n) \]


  \section{Approche récursive}
  \subsection{Analyse de {\texttt{MistFonction3}}}
  Pour trouver toutes les combinaisons possibles de \texttt{target} en utilisant les 
  pièces \texttt{pieces}, chaque appel récursif devra tenter de reconstituer 
  \texttt{target}. Pour cela, chaque combinaison sera recalculée plusieurs fois, 
  puisque pour un caractère donnée de \texttt{target}, on épuises les 
  possibilités de \texttt{pieces}. Si \texttt{target} a une longeur $n$ 
  et \texttt{pieces} a une longueur $m$, la complexité temporelle sera alors :
  \[ O(m^n) \]




  \subsection{Analyse de {\texttt{MistFonction3}}}
  Pour vérifier \textit{récursivement}   
  si une combinaison d'éléments dans un tableau d'options \texttt{options} 
  permet d'engendrer l'entier \texttt{target}, il faudrait évaluer 
  récursivement toutes les combinaisons possibles. Pour chaque élément 
  de \texttt{options}, deux possibilités s'offre à nous : inclure l'entrée à l'index 
  courant dans la somme ou ne pas l'inclure. Cela génère $2^m$ appels récursifs 
  où $m$ est la quantité d'éléments dans les tableau original \texttt{options}.   
  La complexité serait donc :
  \[ O(2^m) \]
  Pour une \texttt{target} \texttt{T}, un tableau d'options \texttt{[A, B, C]} 
  (\texttt{A}, \texttt{B} t \texttt{C} sont des entiers et peuvent prendre n'importe quelle 
  valeur pour cet exemple) et un 
  index $i$, nous avons les appels récursifs suivants. 


    \dirtree{%
    .1 \( f(\texttt{T, [A, B, C], 0}) \).
    .2 Inclure A: \( f(\texttt{T-A, [A, B, C], 1}) \).
    .3 Inclure B: \( f(\texttt{T-A-B, [A, B, C], 2}) \).
    .4 Inclure C: \( f(\texttt{T-A-B-C, [A, B, C], 3}) \).
    .4 Exclure C: \( f(\texttt{T-A-B, [A, B, C], 3}) \).
    .3 Exclure B: \( f(\texttt{T-A, [A, B, C], 2}) \).
    .4 Inclure C: \( f(\texttt{T-A-C, [A, B, C], 3}) \).
    .4 Exclure C: \( f(\texttt{T-A, [A, B, C], 3}) \).
    .2 Exclure A: \( f(\texttt{T, [A, B, C], 1}) \).
    .3 Inclure B: \( f(\texttt{T-B, [A, B, C], 2}) \).
    .4 Inclure C: \( f(\texttt{T-B-C, [A, B, C], 3}) \).
    .4 Exclure C: \( f(\texttt{T-B, [A, B, C], 3}) \).
    .3 Exclure B: \( f(\texttt{T, [A, B, C], 2}) \).
    .4 Inclure C: \( f(\texttt{T-C, [A, B, C], 3}) \).
    .4 Exclure C: \( f(\texttt{T, [A, B, C], 3}) \).
    }

    \chapter{Pile simple}
    \begin{Exercice}{(3 pts)}{}
      Écrire une classe « ArrayStack » qui fonctionne à l'aide d'un tableau 
      générique (n’oubliez pas que vous
      devez également implémenter son interface appelée « Stack »). 
      Le nombre maximal d’éléments est de
      100. Ce sont les méthodes que la pile simple devrait 
      pouvoir exécuter :
    \end{Exercice}


  \begin{table}[H]

    \begin{center}
      \renewcommand{\arraystretch}{1.5}
      \fontfamily{flr}\selectfont
      \footnotesize
          \begin{tabular}{p{2cm} p{4cm} p{4cm}}
          \arrayrulecolor{blue}\hline
          \rowcolor{draculawhite-background}
          \textcolor{myb}{Fonction} & 
          \textcolor{myb}{Signature} & 
          \textcolor{myb}{Détails}  
          \\
          \hline
          \arrayrulecolor{black}
          Push   & 
          \texttt{public void push(E e)}  
                 &
          Ajoute un élément sur la pile 
          \\
          \hline
          Pop   &  
          \texttt{public E pop()}  
                &
          Retire le dernier élément sur la pile et le renvoie
          \\
          \hline
          Top  &
          \texttt{public E top()} 
               &
          Renvoie le dernier élément sur la pile
          \\ 
          \hline
          Size & 
          \texttt{public int size()}  
               &
          Renvoie la longueur de la pile
          \\
          \hline 
          Is Empty & 
          \texttt{public boolean isEmpty()}  
                   &
          Vérifie si la pile est vide
          \\
          \hline 
          To String & 
          \texttt{public String toString()}  
                    &
          Produit une représentation en chaîne des éléments 
          de la pile classés de haut en bas. 
          \\ 
          \hline
          \end{tabular}
  \end{center}
  \end{table}
  \vspace{-1em}

    \begin{lstlisting}[style=JavaDraculaWhite]
public class ArrayStack<E> implements Stack<E> {
  private static final int NombreElementsMaximal = 100;

    private E[] Pile; // utilisation d'un simple tableau pour l'implementation
    
    /* Le haut de la pile a comme index -1 lorsque le tableau  est vide 
       0 lorsque le tableau a un element, 1 lorsque le tableau a 2 element, 
       etc. */
    private int dessus = -1; 

    // Constructeur
    public ArrayStack() {
        Pile = (E[]) new Object[NombreElementsMaximal]; 
        // safe cast; compiler may give warning
    }

    // Implemente les methodes de l'infercae Stack.
    @Override
    public void push(E e) throws IllegalStateException {
        if (size() == NombreElementsMaximal) throw 
          new IllegalStateException("La pile a atteint le nombre maximal d'éléments");

        //incremente l'index de dessus de pile
        this.dessus++; 
        Pile[dessus] = e; // increment t before storing new item
    }

    @Override
    public E pop() {
        if (isEmpty()) return null;
        E elementRetourne = Pile[dessus];
        //Efface artificiellement l'element
        Pile[dessus] = null;
        // Decremente pour avoir un index de dessus valide
        dessus--;
        return elementRetourne;
    }

    @Override
    public E top() {
        if (isEmpty()) return null;
        return Pile[dessus];
    }

    @Override
    public int size() {
        return (dessus + 1);
    }

    @Override
    public boolean isEmpty() {
      // Lorsque le dessus a l'index 1, on sait que la pile est vide
        return (dessus == -1);
    }

    @Override 
    public boolean isFull() {
       // Lorsque la pile est pleine, le dessus a l'index 99
      return (dessus == 99);
    }

    @Override
    public String toString() {
        StringBuilder sb = new StringBuilder("Contenu de la pile: [");
        for (int i = dessus; i >= 0; i--) {
            sb.append(Pile[i]);
            if (i > 0) sb.append(", ");
        }
        sb.append("]");
        return sb.toString();
    }
  
}

public interface Stack<E> {
    void push(E e);
    E pop();
    E top();
    int size();
    boolean isEmpty();
    String toString();
        }
    \end{lstlisting}
    \chapter{Pile Double}
    \vspace{-2em}

    \begin{Exercice}{(4 pts)}{}
        Écrire une classe « ArrayDoubleStack » qui implémente 2 piles
        dans un même tableau générique (n’oubliez
        pas que vous devez également implémenter son interface appelée 
        « DoubleStack »). Le nombre maximal d'éléments (longueur pile 1 + 
        longueur pile 2) est de 100.
        Cette classe doit avoir toutes les fonctions ci-dessus pour chaque pile,
        avec pour seule différence un
        booléen « one » qui indique si l'on traite les éléments à la 1re
        ou 2e pile.       
    \end{Exercice}


  \begin{table}[H]

    \begin{center}
      \renewcommand{\arraystretch}{1.5}
      \fontfamily{flr}\selectfont
      \footnotesize
          \begin{tabular}{p{2cm} p{3cm} p{4cm}}
          \arrayrulecolor{blue}\hline
          \rowcolor{draculawhite-background}
          \textcolor{myb}{Fonction} & 
          \textcolor{myb}{Signature} & 
          \textcolor{myb}{Détails}  
          \\
          \hline
          \arrayrulecolor{black}
          Push   & 
          \texttt{public boolean push(boolean one, E e)}  
                 &
          Ajoute un élément sur la pile et 
          renvoie vrai. Renvoie faux si ce n'est pas possible.
          \\
          \hline
          Pop   &  
          \texttt{public E pop(boolean one)}  
                &
          Retire le dernier élément sur la pile et le renvoie.
          \\
          \hline
          Top  &
          \texttt{public E top(boolean one)} 
               &
          Renvoie le dernier élément sur la pile.
          \\ 
          \hline
          Size & 
          \texttt{public int size(boolean one)}  
               &
          Renvoie la longueur de la pile.
          \\
          \hline 
          Is Full & 
          \texttt{public boolean isFull()}  
                   &
          Vérifie si la pile est pleine.
          \\
          \hline 
          Print & 
          \texttt{public void print()}  
                    &
          Imprime le contenu des 2 piles.
          \\ 
          \hline
          \end{tabular}
  \end{center}
  \end{table}

    \begin{lstlisting}[style=JavaDraculaWhite]
 interface DoubleStack<E> {
    boolean push(boolean one, E element);
    E pop(boolean one);
    E top(boolean one);
    int size(boolean one);
    boolean isFull();
    void print();
}      
public class ArrayDoubleStack<E> implements DoubleStack<E> {
    private E[] doublePile;
    private int nombreMaximalElements = 100;
    private int dessus1;
    private int dessus2;

    public ArrayDoubleStack() {
        doublePile = (E[]) new Object[nombreMaximalElements];
        // Le dessus de la pile 1 se trouve du cote "gauche" du tableau
        dessus1 = -1;

        // Le dessus de la pile 2 se trouve du cote "droit" du tableau
        dessus2 = nombreMaximalElements;
    }

    /* La pile est pleine lorsque l'index de pile 1 est -1 (index initial) + 50 + 1= 50 et 
     * l'index de pile 2 est 100 (index initial) - 50 = 50*/
    public boolean isFull() {
        return (dessus1 + 1 == dessus2);
    }

    /* La methode verifie d'abord si la double pile est pleine et ajoute 
     * l'element au dessus du bon cote de la doublePile en fonction 
     * de l'argument fourni (true ou false) */
    public boolean push(boolean one, E element) {
        if (isFull()) {
            return false;
        }
        if (one) {
            /* On increment le dessus de pile 1 et met 
             * l'element au 'nouveau' desuss1 */
              doublePile[++dessus1] = element;
        } else {
            /* Ou alors On decrement le dessus de pile 2 et met 
             * l'element au 'nouveau' desuss2 */
            doublePile[--dessus2] = element;
        }
        return true;
    }
    
    public E pop(boolean one) {
        if (one) {
            // Verifie que la pile contient au moin un element
            if (dessus1 >= 0) {
                /* Decrement l'index de dessus, efface le contenu de dessus 
                et retourne la valeur qui etait au dessus1 */ 
                E reponse = doublePile[dessus1];
                doublePile[dessus1] = null; 
                dessus1--; 
                return reponse;
            }
        } else if (!one && dessus2 < nombreMaximalElements) {
            E reponse = doublePile[dessus2]; 
            doublePile[dessus2] = null;
            dessus2++;
            return reponse;  
            
        }
          else {
        // Si aucune des pile ne contient d'element 
            throw new IllegalStateException(
                "La double pile est vide, vous ne pouvez pas obtenir d'element. ")
        }
    }

    /* En fonction du boolean fourni, on verifie si la pile 
     * est plein et retourne l'element du dessus1 ou dessus2 */
    public E top(boolean one) {
        if (one) {
            if (dessus1 >= 0) {
                return doublePile[dessus1];
            }
        } else if (!one && dessus2 < nombreMaximalElements)  {
            return doublePile[dessus2];
        }
          else {
              throw new IllegalStateException(
                  "La double pile est vide, vous ne pouvez pas regarder l'element. ")
        }
    }
    
    /* En fonction du boolean fourni, on retourne la valeur   
     * dessus1 + 1 (taille pile 1) ou nbMaximalElements - dessus (taille pile 2) */
    public int size(boolean one) {
        if (one) {
            return dessus1 + 1;
        } else {
            return nombreMaximalElements - dessus2;
        }
    }


    public void print() {
        System.out.println("Pile 1:");
        for (int i = 0; i <= dessus1; i++) {
            System.out.println(doublePile[i]);
        }
        System.out.println("Pile 2:");
        for (int i = nombreMaximalElements - 1; i >= dessus2; i--) {
            System.out.println(doublePile[i]);
        }
    }
}
    \end{lstlisting}

    \chapter{Pile spéciale}

    \begin{Exercice}{(3 pts)}{}
      Écrire une classe « SpecialArrayStack » qui fonctionne à l'aide 
      d'un tableau générique (N’oubliez pas que
      vous devez également implémenter son interface appelée « SpecialStack »).
      Le nombre maximal d’éléments est de 100. La pile spéciale devrait 
      implémenter une méthode getMax() qui devrait renvoyer
      l’élément maximum stocké dans la pile spéciale en O(1) temps et 
      O(1) espace supplémentaire, en plus des
      mêmes méthodes qui ont été implémentées dans «Pile simple».
    \end{Exercice}

    \begin{lstlisting}[style=JavaDraculaWhite]
 /* Seules les classes qui extends l'interface Java Comparable peuvent être stockes dans 
 * la SpecialArrayStack. Notre methode push(E e) a besoin de comparer l'element qu'on est sur 
 * le point d'ajouter (e) a l'element Maximal  Maximaux[prochainMaximum]. 
 * Puisqu'on ne connait pas a l'avance le tyupe des objets stockes dans la 
 * SpecialArrayStack, il faut utiliser le type generique E */
public class SpecialArrayStack<E extends Comparable<E>> implements Stack<E> {   

    private static final int NombreElementsMaximal = 100;
    private E[] Pile;
    private E[] Maximaux; // Tableau pour stocker les éléments maximaux
    private int dessus = -1;
    private int prochainMaximum = -1; // Index pour suivre le prochain élément maximal


    // Constructeur
    public SpecialArrayStack() {
        Pile = (E[]) new Object[NombreElementsMaximal];
        // Initialisation du tableau des maximaux
        Maximaux = (E[]) new Object[NombreElementsMaximal]; 
    }

    // Implémente les méthodes de l'interface Stack...
    @Override
    public void push(E e) throws IllegalStateException {
        if (isFull()) {
            throw new 
            IllegalStateException("La pile a atteint le nombre maximal d'éléments");
        }
        
        if (isEmpty() || e.compareTo(getMax()) >= 0) {
            prochainMaximum++; 
            // Ajoute le nouveau maximum au tableau des maximaux
            Maximaux[prochainMaximum] = e;
        }
        dessus++;
        Pile[dessus] = e; // Ajoute l'élément à la pile
    }

    @Override
     public E pop() {
            if (isEmpty()) {
                throw new 
                IllegalStateException("La pile est vide, vous ne pouvez pas retirer d'élément.");
            }
            
            E elementRetourne = Pile[dessus];
            Pile[dessus] = null; // Efface artificiellement l'élément
            dessus--;
            
            if (elementRetourne.equals(getMax())) {
            // Retire le maximum actuel du tableau des maximaux
                Maximaux[prochainMaximum] = null; 
                prochainMaximum--;
            }
            
            return elementRetourne;
        }

    @Override
    public E top() {
        if (isEmpty()) return null;
        return Pile[dessus];
    }

    @Override
    public int size() {
        return (dessus + 1);
    }

    @Override
    public boolean isEmpty() {
      // Lorsque le dessus a l'index 1, on sait que la pile est vide
        return (dessus == -1);
    }

    @Override 
    public boolean isFull() {
        return (dessus == NombreElementsMaximal - 1);
    }


    @Override
    public String toString() {
        StringBuilder sb = new StringBuilder("Contenu de la pile: [");
        for (int i = dessus; i >= 0; i--) {
            sb.append(Pile[i]);
            if (i > 0) sb.append(", ");
        }
        sb.append("]");
        return sb.toString();
    }

    public E getMax() {
        if (prochainMaximum == -1) {
            throw new IllegalStateException("La pile est vide, il n'y a pas d'élément maximal.");
        }
        return Maximaux[prochainMaximum];
    }  
}      
    \end{lstlisting}














  %Retourne une liste de listes de ch. char. 
  %Prend une chain char. et un Liste de ch. char. 

  % Crée un tableau de la taille de la ch. char target 
  % Chaque élément du tableau table possède maintenant ArrayList
  % ajoute au début du tableau table une nouvelle ArrayList









\end{document}
